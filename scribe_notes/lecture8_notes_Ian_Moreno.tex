\documentclass{article}
\usepackage{amsmath}
\usepackage{graphicx}
%\addtolength{\textheight}{+ .1\textheight}
\title{CS428 Graphics Lecture Monday 03/04/2019}
\author{Ian Moreno}
%\date{03/04/2019}

\begin{document}
	\maketitle 
	
	\section{Rotations}

	\begin{itemize}
		\item  NOTE: 3x3 matrices have 9 numbers but only 3 are independent
	\end{itemize}
	\begin{table}[h]
		\begin{center}
			{\small Euler Angles Have Singularities (Inverse problem cannot always be solved, however)}
			\linebreak
		{\renewcommand{\arraystretch}{2}%
			\begin{tabular}{|l|l|}
				\hline
				ZYZ & Refers to rotation angle about current frame \\
				\hline
				RPY & Refers to rotation angle about fixed frame \\
				\hline
				Angle And Axis & A general 3D $\vec{v}$ and angle $\alpha$ \\
				\hline
		\end{tabular}} \quad
		\end{center}
	\end{table}
	
	\section{Unit Quaternions}
	\begin{itemize}
		\item  Using Quaternions, the inverse problem can \underline{always} be solved.
		\item Unit Quaternions represented by 4 numbers
	\end{itemize}
	
	$Q = ($\space$\zeta, \vec{\varepsilon}$\space$)$\space,\space$\zeta = \cos(\frac{v}{2})$,\space$\varepsilon = sin(\frac{v}{2})\vec{\gamma}$
	\begin{center}$v =$\space angle of rotation, $\gamma =$\space axis of rotation \end{center}

	$\vec{\varepsilon} = \begin{bmatrix}\varepsilon_x \\ \varepsilon_y \\ \varepsilon_z \end{bmatrix} $
	\begin{center}
	Assume $\vec{\gamma}$ is a unit vector then $\vec{\varepsilon} = sin(\frac{v}{2})\vec{\gamma}$
	\newline
	$||\vec{\gamma}||^2 = 1$
	\newline
	$\gamma_x^2 + \gamma_y^2 + \gamma_x^2 = 1$
	\newline
	\newline
	$\vec{\varepsilon}  = sin(\frac{v}{2})\vec{\gamma}$,\space$||\vec{\varepsilon}||^2 = \varepsilon_x^2 + \varepsilon_y^2 + \varepsilon_z^2 = sin^2(\frac{v}{2})(\gamma_x^2 + \gamma_y^2 + \gamma_z^2) = sin^2(\frac{v}{2})$
	\newline
	\newline
	$\zeta^2 + ||\vec{\varepsilon}||^2 = cos^2(\frac{v}{2}) + sin^2(\frac{v}{2}) = 1$
	\newline
	\newline
	$Q = ($\space$\zeta, \vec{\varepsilon}$\space)\space\space\space\fbox{$\zeta^2 + \varepsilon_x^2 + \varepsilon_y^2 + \varepsilon_z^2 = 1$} $\leftarrow$ Unit Quaternion
	
	\end{center}

\subsection{Rotation Matrix With Quaternion}
\begin{center}
	$R($\space$\zeta, \vec{\varepsilon}$\space$) = 
	\begin{bmatrix}
	2(\zeta^2 + \varepsilon_x^2)-1 & 2(\varepsilon_x\varepsilon_y - \zeta\varepsilon_z & 2(\varepsilon_x\varepsilon_z + \zeta\varepsilon_y) \\
	2(\varepsilon_x\varepsilon_y + \zeta\varepsilon_z) & 2(\zeta^2 + \varepsilon_y^2)-1 & 2(\varepsilon_y\varepsilon_z - \zeta\varepsilon_x) \\
	2(\varepsilon_x\varepsilon_z - \zeta\varepsilon_y) & 2(\varepsilon_y\varepsilon_z + \zeta\varepsilon_x) & 2(\zeta^2 + \varepsilon_z^2)-1
	\end{bmatrix}$
	\newline
	(if you represent the rotation as a quaternion)
\end{center}

\section{Inverse Problem: Construct rotation matrix and convert it to a quaternion}
$R = 
\begin{bmatrix}
\gamma_{11} & \gamma_{12} & \gamma_{13} \\
\gamma_{21} & \gamma_{22} & \gamma_{23} \\
\gamma_{31} & \gamma_{32} & \gamma_{33} 
\end{bmatrix}$
\newline
\newline
$\zeta = \frac{1}{2}\sqrt{\gamma_{11} + \gamma_{22} + \gamma_{33} +1}$
\newline
\newline
$\vec{\varepsilon} = \frac{1}{2} 
\begin{bmatrix}
sin(\gamma_{32}-\gamma_{23})\sqrt{\gamma_{11} - \gamma_{22} - \gamma_{33} +1} \\
sin(\gamma_{13}-\gamma_{31})\sqrt{\gamma_{22} - \gamma_{33} - \gamma_{11} +1} \\ sin(\gamma_{21}-\gamma_{12})\sqrt{\gamma_{33} - \gamma_{11} - \gamma_{22} +1}
\end{bmatrix} = 
\begin{bmatrix}
\varepsilon_x \\
\varepsilon_y \\
\varepsilon_z
\end{bmatrix}$
\newline
- Note: $sin(x) = 1, $\space for\space$ x >= 0, $\space and\space$ sin(x) = -1, $\space for\space$ x < 0 $
\newline
\newline
We always use quaternions in graphics pipeline
\newline
\newline
$Q \equiv ($\space$\zeta, \vec{\varepsilon}$\space$) \equiv R($\space$\zeta, \vec{\varepsilon}$\space$)$
\newline
$RR^T = I \rightarrow R^{-1} = R^T$
\newline
$R^{-1}($\space$\zeta, \vec{\varepsilon}$\space$) \equiv Q^{-1}($\space$\zeta, \vec{\varepsilon}$\space$) \equiv ($\space$\zeta, -\vec{\varepsilon}$\space$)$
\newline
\newline
$Q_1($\space$\zeta_1, \vec{\varepsilon}_1) \equiv R_1($\space$\zeta_1, \vec{\varepsilon}_1$\space$)$
\newline
$Q_2$\space$\zeta_2, \vec{\varepsilon}_2) \equiv R_2($\space$\zeta_2, \vec{\varepsilon}_2$\space$)$
\newline
\newline
$R_1*R_2 \equiv R_1R_2$
\newline
\newline
\fbox{$Q_1*Q_2 \equiv ($\space$\zeta_1\zeta_2 - \varepsilon_1^T\varepsilon_2)$,\space\space$\zeta_1\vec{\varepsilon}_2 + \zeta_2\vec{\varepsilon}_1 + \vec{\varepsilon}_1*\vec{\varepsilon}_2)$}
\linebreak
\newline
Suppose  $Q_2 \equiv Q_1^{-1} \implies Q + Q_2 = Q * Q_1^{-1} = I = ( 1, \vec{0})$
\newline
$Q \equiv ($\space$\zeta, \vec{\varepsilon}$\space$),$\space$Q^{-1} \equiv ($\space$\zeta, -\vec{\varepsilon}$\space$)$
\newline
So $\zeta_1\zeta_2 - \varepsilon_1^T\varepsilon_2 = \zeta^2 + \varepsilon^T\varepsilon = \zeta^2 + \varepsilon_x^2 + \varepsilon_y^2 + \varepsilon_z^2 = 1$ (by definition of Quaternion)
\newline
$\zeta_1\vec{\varepsilon}_2 + \zeta_2\vec{\varepsilon}_1 + \vec{\varepsilon}_1*\vec{\varepsilon}_2 = - \zeta\vec{\varepsilon} + \zeta\vec{\varepsilon} - \vec{\varepsilon}*\vec{\varepsilon} = 0$

\section{Homogeneous Transformations}
\includegraphics[width=1.0\textwidth]{graph}
\newline
\newline
$p^0 = O_1^0 + R_1^0P^1$ 
\newline
Instead of using matrix vector multiplication and vector-vector addtition we just use matrix-vector multiplication

See shader.vs in the Transformations lecture
\newline
\newline
$\hat{P} = 
\begin{bmatrix}
P_{[3x1]}^1 \\
1
\end{bmatrix}_{[4x1]}$,\space$A_1^0 = \begin{bmatrix}
R_{1 [3x3]}^0 & O_{1 [3x1]}^0 \\
0^T & 1_{[1x1]}\\
\varepsilon_z
\end{bmatrix}_{[4x4]}$,\space$0^T = \begin{bmatrix}
0 & 0 & 0
\end{bmatrix}_{[1x3]}$
\newline
\newline
$P^0 = O_1^0 + R_1^0P^1$
\newline
${R_1^{0}}^{-1} = R_1^{0 T} \implies {R_1^{0}}^{-1}P^0 = R_1^{0 T}P^0 = R_1^{0 T}O_1^0 + R_1^{0 T}R_1^0P_1$,\space$R_1^{0 T}R_1^0P_1 =  P^1$
$R_1^{0 T}P^0 = R_1^{0 T}O_1^0 + P^1 \implies P^1 = -R_1^{0 T}O_1^0 + R_1^{0 T}P^0$
\newline
\newline
$A_0^1 = \begin{bmatrix}
R_1^{0 T} & -R_1^{0 T}O_1^0 \\
0^T & 1
\end{bmatrix} * \begin{bmatrix}
P^0 \\
1
\end{bmatrix}_{(\hat{P}^0)} = \begin{bmatrix}
P^1 \\
1
\end{bmatrix}_{(\hat{P}^1)}$
\newline
\newline
$\hat{P}^1 = A_0^1\hat{P}^0$,\space$\hat{P}^0 = A_1^0\hat{P}^1$
$\hat{P}^1 = A_0^1A_1^0\hat{P}^0 \implies A_0^1A_1^0 = I \implies A_0^1 = {A_1^0}^{-1}$
\newline
\newline
$A_1^0 = \begin{bmatrix}
R_1^0 & O_1^0 \\
0^T & 1
\end{bmatrix}$,\space$A_0 = \begin{bmatrix}
R_1^{0T} & -R_1^{0 T}O_1^0 \\
0^T & 1
\end{bmatrix}$
\newline
\newline
NOTE\space$A_0^1 \not= A_1^{0 T}$ but $R_0^1 = R_1^{0 T}$

\section{Manipulator Arm}
\includegraphics[width=1.0\textwidth]{gripper}
\end{document}